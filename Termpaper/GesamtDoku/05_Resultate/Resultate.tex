% !TEX root = ../Dokumentation.tex
\section{Auswertung / Resultat}
In der vorliegenden Arbeit ist ein Temperaturüberwachungssystem erstellt und getestet worden, welches eine zuverlässige Überwachung einer Region in einem Serverraum oder kleinerem Rechenzenter ermöglicht.\\
Der Aufbau ist dabei mit steckbaren Komponenten realisiert und erfordert daher vom Wartungsteam keine tieferen Kentnisse in der Elektronik. Durch die Möglichkeit, vier Sensoren anschliessen zu können, mit den im Kapitel \glqq{}Verwendete Hardware\grqq{} modifikationen auf acht Sensoren erweiterbar, kann einerseits eine Redundaz erzeut werden, falls ein Sensor defekt ist. Weiter ist es möglich, mit einem Raspberry Pi mehrere Messpunkte zu überwachen, unter Berücksichtigung, dass bei grösseren Distanzen die Messresultate verfälscht werden. Daher sind die verwendeten Kabel nur dann durch längere zu ersetzen, wenn entsprechende Tests die Zuverlässigkeit des Systems sicherstellen. Angehängte Sensoren werden automatisch erkannt und auch geloggt, die gleichen Eigenschaften gelten, bei Ausfall oder Entfernen von Sensoren.\\
Die Software läuft als Dienst, sobald das Gerät gestartet ist und verbraucht dank langsamer Tacktung nur wenig Energie. Der modulare Aufbau mit kleinen Klassen bietet eine leichte Wartbarkeit, der Austausch einzelner Elemente lässt sich leicht gestalten. Die Konfiguration des Gerätes verlangt abenfalls nicht sehr Tiefe Informatik- Kenntnisse, so könnte diese Lösung auch im Heimgebrauch zum Einsatz kommen.
\newpage