% !TEX root = ../Dokumentation.tex
\section{Auswertung / Resultat}
In der vorliegenden Arbeit ist ein Temperaturüberwachungssystem erstellt und getestet worden, welches eine zuverlässige Überwachung einer Region in einem Serverraum oder kleinerem Rechenzentrum ermöglicht.\\
Der Aufbau ist dabei mit steckbaren Komponenten realisiert und erfordert vom Wartungsteam keine tieferen Kenntnisse in der Elektronik. Durch die Möglichkeit, vier Sensoren anschliessen zu können, mit der im Kapitel \glqq{}Verwendete Hardware\grqq{} Modifikationen auf acht Sensoren erweiterbar, kann bei einem defekten Sensor eine Redundanz erzeugt werden. Weiter ist es möglich, mit einem Raspberry Pi, unter Berücksichtigung das bei grösseren Distanzen die Messresultate verfälscht werden, mehrere Messpunkte zu überwachen. Daher sind die verwendeten Kabel nur dann durch längere zu ersetzen, wenn entsprechende Tests die Zuverlässigkeit des Systems sicherstellen. Angehängte Sensoren werden automatisch erkannt und auch geloggt, die gleichen Eigenschaften gelten, bei Ausfall oder Entfernung von Sensoren.\\
Die Software läuft als Dienst, sobald das Gerät gestartet ist und verbraucht dank langsamer Taktung nur wenig Energie. Der modulare Aufbau mit kleinen Klassen bietet eine leichte Wartbarkeit, der Austausch einzelner Elemente lässt sich dadurch leicht gestalten. Die Konfiguration des Gerätes verlangt ebenfalls nicht sehr Tiefe Informatik- Kenntnisse, so könnte diese Lösung auch im Heimgebrauch zum Einsatz kommen.
\newpage