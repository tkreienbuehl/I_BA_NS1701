% !TEX root = ../Dokumentation.tex
\section*{Abstract}
\addcontentsline{toc}{section}{Abstract}
Im vorliegenden Termpaper wird die Realisierung einer Temperaturüberwachung mit Temperatursensoren durchgeführt. Dieses Projekt  wurde im Rahmen des Moduls \grqq{}Network, Cloud \& Services\glqq{} , auf Grund eines Vorfalls im Enterprise Lab, zur Umsetzung freigegeben. Bei diesem Geschehnis kam es,auf Grund einer nicht mehr funktionierenden Kühlung, zu einer Überhitzung im Enterprise Lab.
Für die Umsetzung wurde ein Raspberry Pi, eine Basisplatine sowie 8 Sensorplatinen verwendet. Die Software misst mittels angeschlossenen Sensoren die aktuelle Temperatur. Die daraus berechneten Sensordaten werden dabei in eine XML-Datei geladen. Sobald die Temperatur in einen kritischen Bereich gerät, wird umgehend eine Mailbenachrichtigung an das Wartungsteam gesendet. Ein zusätzliches Feature ist die webbasierte Anzeige der aktiven Sensoren, welche via JavaScrip periodisch neu geladen werden.
 
\clearpage