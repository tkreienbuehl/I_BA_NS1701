% !TEX root = ../Dokumentation.tex
\section*{Abstract}
\addcontentsline{toc}{section}{Abstract}
Im vorliegenden Termpaper wird die Realisierung einer Temperaturüberwachung mit Temperatursensoren durchgeführt. Dieses Projekt  wurde im Rahmen des Moduls \grqq{}Network, Cloud \& Services\glqq{} , durch einen Vorfall im Enterprise Lab, zur Umsetzung freigegeben. Bei diesem Geschehnis kam es, auf Grund einer nicht mehr funktionierenden Kühlung, zu einer Überhitzung im Enterprise Lab. Dies hat dazu geführt, dass die Hardware Schäden erlitt und sämtliche Services nicht mehr liefen.
Zu Beginn wurden alle notwendigen Anforderungen eines Temperaturüberwachungssystems aufgenommen. Ebenfalls musste eine gründliche Technologierecherche der benötigten Hardware vorgenommen werden, damit eine erfolgreiche Realisierung möglich ist.
Für die Umsetzung wurden ein Raspberry Pi, eine Basisplatine sowie 8 Sensorplatinen verwendet. Die Software misst die aktuellen Temperaturen der Sensoren, welche über die elektrischen Anschlusspunkte, auf der Basisplatine angeschlossen sind. Die daraus berechneten Sensordaten werden dabei in eine XML-Datei gespeichert. Sobald die Temperatur in einen kritischen Bereich steigt, wird umgehend eine Mailbenachrichtigung an das Wartungsteam gesendet. Ein zusätzliches Feature ist die webbasierte Anzeige der aktiven Sensoren, welche via JavaScrip erstellt und periodisch neu geladen werden.
 
\clearpage