% !TEX root = ../Dokumentation.tex
\section{Realisierung}

\subsection{Raspberry Pi}

Folgende Installation \& Grundkonfigurationen müssen zu Beginn am Raspberry Pi vorgenommen werden.

\subsubsection{Software}
Die 32GB grosse SD-Karte welche man für das Raspberry benötigt, wird zu Beginn formatiert um danach eine saubere Installation des Betriebssystems durchzuführen. Als OS wird die Version 2017-04-10 raspbian-jessie verwendet, dieses Betriebssystem wird von diversen Webseiten und Anleitungen empfohlen. Es hat eine graphische Oberfläche und bietet bereits vorinstallierte Programme.

\subsubsection{Konfiguration}
Nach der Installation der Software kann das Raspberry mit Maus und Tastatur an einen Monitor angeschlossen werden. Sobald das Raspberry mit Strom versorgt wird, startet der Boot-Vorgang.
Nun müssen alle wichtigen Grundkonfigurationen vorgenommen werden, aktuell auf dem vorliegenden Gerät folgendermassen umgesetzt.
\begin{itemize}
	\item Tastatur: Deutsch(Schweiz)
	\item Username: pi
	\item Passwort: temperatur2017
	\item Inteface SPI: enable
	\item Netzwerk: aktuell ist die IP-Adresse fix mit 10.25.0.7 konfiguriert. Eine fixe Zuordung der IP Adresse wird empfohlen, um das Pi zuverlässig über ssh erreichen zu können. Dies kann ebenfalls über eine DHCP Reservierung mittels MAC-Binding erreicht werden.
\end{itemize}
Das Raspberry Pi ist in der Grundkofiguration sehr offen eingestellt, so hat der Standardbenutzer root- Rechte ohne das Passwort eingeben zu müssen. Dies, um schnell zu Ergebnissen bei kleineren Versuchen zu kommen. Die GPIO- Ports sowie die meisten I/O-Files müssen immer mit root- Rechten angesprochen werden. Die entsprechenden Einträge dazu befinden sich in der Datei \code{/etc/sudoers} und der dazu gehörende Eintrag heisst \code{pi ALL=(ALL) NOPASSWD: ALL}. Unbedingt beachtet werden muss, dass eine fehlerhafte Änderung dieser Date dazu führen kann, dass keine root- Befehle mehr ausgeführt werden können. Die sichere Konfiguration des Raspberry Pi ist ausserhalb des Scope dieser Arbeit, es soll nur auf gewisse Risiken hingewiesen werden.

\subsubsection{Webserver}
Auf dem Raspberry wird zusätzlich ein Webserver bereitgestellt, damit die Temperaturausgabe direkt im lokalen Netz betrachtet werden kann. Als Webserver wurde Apache2 installiert. Die Webseite selber wird dann im Stammverzeichnins /var/www/html gespeichert und ist über die IP oder, wenn auf dem DNS-Server eingetragen auch über den Namen erreichbar. Aus Sicherheitsgründen sollte die Webseite statisch gehalten sein und nur über das Intranet erreicht werden können, wenn nicht zusätzliche Sicherheitsmassnahmen getroffen werden. 

\subsection{Softwaredesign}

Das Klassendiagramm (Abbildung \ref{fig:classdia}) bietet eine detaillierte Übersicht über die Softwarelösung. Im Zentrum steht dabei die Klasse \code{Controller}, welche für die Sensoren, das XML-Handling sowie der Ausgabe benötigter Meldungen zuständig ist. Die digitalen Sensor- Rohdaten werden von der Klasse \code{SPIDataHandler} ermittelt, und vom \code{RawDigitalValueServer} bereitgestellt.

\begin{figure}[H]%Position festigen
\centering
\includegraphics[width=1\textwidth]{Images/Klassendiagramm.png}
\caption{Klassendiagramm Software}
\label{fig:classdia}
\end{figure}

Die Klasse \code{SensorWatchDog} ist dafür zuständig um die angeschlossenen Sensoren zu prüfen, wird ein neuer Sensor angeschlossen, so wird dies dem Controller mittels der Methode \code{reportSensorUp} bekannt gegeben. Der Controller erzeugt dann, einen neuen \code{SensorHandler} und startet einen Thread. Die Klasse \code{SensorHandler} überprüft und berechnet nun laufend die Temperaturwerte und rapportiert diese über den \code{SensorObserver} dem \code{Controller}. Der Wert des Sensors wird schlussendlich mittels \code{XMLHandler} in das File sensordata.xml geschrieben. Dieses File wird für die Ausgabe auf der Webseite verwendet.
Ebenfalls gibt es eine Logger-Klasse welche alle Meldungen mit einem Zeitstempel ausgibt. Die Ausgabe erfolgt auf der Konsole und kann beim Aufruf des Programmes entsprechend in ein Logfile umgelenkt werden. So kann genau überprüft werden, wann ein Sensor angeschlossen oder entfernt, sowie ob zum richtigen Zeitpunkt ein Mail versendet worden ist. 

\subsection{Verwendete Hardware}

<<<<<<< HEAD
Die in Abbildung \ref{fig:plate} dargestellte Baseboard, des Herstellers Linkerkit, bietet eine universelle Plattform um verschiedene Aktoren anschliessen zu können. Für die verwendeten, analogen Temperatursensoren stehen vier Anschlüsse zur Verfügung, wobei jeder Anschluss zwei analoge Kanäle abdeckt. Weiter beitet das verwendete Board je einen UART- und $I^2C$ Anschluss, sowie zwölf GPIO- Eingänge. Die Speisespannung kann mittels einem Jumper auf 3.3V oder 5V gestellt werden. 
=======
Die in Abbildung \ref{fig:plate} dargestellte Baseboard, des Herstellers Linkerkit, bietet eine universelle Plattform um verschiedene Aktoren anschliessen zu können. Für die verwendeten, analogen Temperatursensoren stehen vier Anschlüsse zur Verfügung, wobei jeder Anschluss zwei analoge Kanäle abdeckt.
>>>>>>> 0fba2e361bb582d6b96c44d7688b264fbd5f658c

\begin{figure}[H]%Position festigen
\centering
\includegraphics[width=0.5\textwidth]{Images/Basisplatine.png}
\caption{Basisplatine zum Raspberry Pi (Quelle: Datenblatt Basisplatine)}
\label{fig:plate}
\end{figure}

\begin{figure}[H]%Position festigen
\centering
\includegraphics[width=0.25\textwidth]{Images/Sensorplatine.jpg}
\caption{Sensorplatine mit Temperaturfühler (Quelle www.Conrad.ch)}
\label{fig:sensor}
\end{figure}

Ein analoger Temperaturfühler auf einem passenden Board, in Abbildung \ref{fig:sensor} abgebildet, liefert Spannungswerte, abhängig von der Aussentemperatur. Diese Werte werden über den Analog- Digitalwandlerchip MCP3008, der sich auf dem Baseboard befindet in digitale Signale umgewandelt, welche mittels der Formel
\[
	T = 100 \cdot \left( \frac{D \cdot V}{1024} - 0.5 \right) 
\]
in Grad Celsius umgerechnet werden. Dabei entspricht $D$ dem digitalen Eingangswert, $V$ der Speisespannung für den Sensor, in unserem Fall 3.3V und $T$ der resultierenden Temperatur.

\begin{figure}[H]%Position festigen
\centering
\includegraphics[width=0.25\textwidth]{Images/Verbindungskabel.jpg}
\caption{Verbindungskabel (Quelle www.Conrad.ch)}
\label{fig:cable}
\end{figure}

Die Sensoren werden mittels dem in Abbildung \ref{fig:cable} dargestellten Kabel verbunden. Um acht Sensoren anschliessen zu können müssen spezielle Kabel angefertigt werden, da die Sensorplatine (Abbildung \ref{fig:sensor} einen Anschluss nicht verwendet. Das dazu benötigte Schema ist in Abbildung \ref{fig:schema_doppelsensor} visualisiert. Wichtig ist dabei, auf allfällige Störfaktoren zu achten. Sämtliche Verbindungsstellen sollten zu den übrigen Verbindungsstellen abgeschirmt werden. Dieses Layout wurde im Rahmen dieser Arbeit nicht realisiert.

\begin{figure}[H]%Position festigen
\centering
\includegraphics[width=0.25\textwidth]{Images/Schema.png}
\caption{Schema Verbindungskabel für zwei Sensoren}
\label{fig:schema_doppelsensor}
\end{figure}

\subsection{Webdarstellung}
Die Umsetzung der Webseite wurde hauptsächlich mit JavaScript gemacht.
Für die Anzeige der Temperatur wurde das JavaScript Plug-In JustGage verwendet. Diese visuelle Darstellung ermöglicht es, auf einen Blick zu erkennen, wie hoch die aktuelle Umgebungstemperatur ist. Zudem werden die Sensoranzeigen farblich hervorgehoben, sobald die Temperaturen in einen kritischen Zustand geraten. 
Die Werte werden von der Software als XML-File geliefert, in welchem die Temperaturwerte sowie die ID pro Sensor gespeichert sind. Die Webseite wurde so konzipiert, dass dieses XML-File in einem regelmässigen Zyklus gelesen und die Temperaturwerte entsprechend auf der Seite ausgegeben werden. Dieser Punkt gehört mitunter zu einer der Wichtigsten, denn die korrekte Anzeige der Temperaturwerte ist notwendig, um die Kontrolle jederzeit zu gewährleisten. Diese Anforderung kann mit \code {setInterval(function, 1000)} umgesetzt werden. Mit dieser Funktion werden die Temperaturwerte jede Sekunde oder nach Belieben neu vom XML-File gelesen.
Zusätzlich bringt JustGage eine Funktion mit sich, welche es ermöglicht über \code {refresh(val)} dem Sensor die neuen Werte zu übergeben. Wird ein Sensor angeschlossen, so sollte dieser zeitnah in das XML geschrieben und mittels \code {setInterval(function, 1000)} auf der Webseite ausgegeben werden. Auf der Webseite werden alle 8 Sensoren dargestellt. Wenn die Sensoren effektive an den GPIO's angeschlossen sind,  werden diese auf der Webseite als \grqq{}aktiv\glqq{} angezeigt. Im Gegenzug ist der Sensor \grqq{}inaktiv\glqq{} sobald dieser von der Basisplatine entfernt wird.\\
In der Funktion \code {readXMLNode(xml)} werden die Werte via \code{getEelemtsByTagName} aus dem XML gelesen und in ein Array gefüllt. Ist im XML der Sensor nicht vorhanden, so ist dieser entsprechend nicht angeschlossen und der Wert im Array wird mit 0 befüllt.\\

\textbf{Beispielcode der Webseite}\newline 
In dem folgenden Code wird eine Instanz des Sensors erstellt. In der Abbildung \ref{fig:sensor_webpage} ist die Darstellung mit einem aktiven Sensor und einer übergebenen Temperatur ersichtlich. 
\begin{lstlisting}{}
//Erstellen eines Sensors
var s1 = new JustGage({
	id: "s1",
	value: 0,
	min: 0,
	max: 75,
	title: "Sensor 1",
	hideMinMax: true
});
\end{lstlisting}

\begin{figure}[H]%Position festigen
\centering
\includegraphics[width=0.25\textwidth]{Images/SensorWebseite.jpg}
\caption{Anzeige eines einzelnen Sensors}
\label{fig:sensor_webpage}
\end{figure}

Die untenstehende Funktion ist dafür zuständig, alle Sensorwerte aus der XML-Datei zu lesen. Zuerst wird mit einer for-Schleife durch das Array iteriert und jedem Index den Wert 0 zugewiesen. Anschliessend wird für jeden aktiven Sensor, Überprüfung mit \code{value > 0}, der aktuelle Wert dem Array übergeben.
\begin{lstlisting}{}
//Auslesen der Temperaturwerte
function readXMLNode(xml) {
	var xmlDoc = xml.responseXML;
	for (var i = 0, len = valueArray.length; i < len; i++) {
		valueArray[i] = 0;
	}                
	for (var i = 0, len = valueArray.length; i < len; i++) {
		sensorID = xmlDoc.getElementsByTagName("SensorValues")
		[0].getElementsByTagName("Sensor")[i].getAttribute("id");
		value = xmlDoc.getElementsByTagName("SensorValues")
		[0].getElementsByTagName("Sensor")[i].getAttribute("value");
		if (value> 0) {
			valueArray[sensorID] = value;
		}
	}
};
\end{lstlisting}

Im folgenden Codebeispiel wird das XML jede Sekunde neu geladen und über \code{s1.refresh(valueArray[0])} an den Sensor, welcher auf der Webseite angezeigt wird, übergeben. Im Array wird zusätzlich geprüft, ob der Wert null oder 0 ist. Ist dies der Fall so wird auf der Seite inaktiv ausgegben.

\begin{lstlisting}{}
//Zyklische Uebergabe der Temperaturwerte an den Sensor
setInterval(function () {
	loadXmlFile();
	s1.refresh(valueArray[0]);
	if (valueArray[0] != null && valueArray[0] > 0) {
		document.getElementsByClassName('container')[0].
		getElementsByClassName('t1')[0].innerHTML = 'aktiv';
	} else {
		document.getElementsByClassName('container')[0].
		getElementsByClassName('t1')[0].innerHTML = 'inaktiv';
	}
}, 1000);
\end{lstlisting}

\begin{figure}[H]%Position festigen
\centering
\includegraphics[width=0.75\textwidth]{Images/Webseite.png}
\caption{Ausgabe der Webseite}
\label{fig:webpage}
\end{figure}

\section{Versuchsdurchführung}
Testing war und ist während dem gesamten Realisierungsprozesses ein zentrales Thema. Zu Beginn wurde das Softwarekonzept lokal auf dem Notebook erstellt und schrittweise getestet, mittels Konsolenausgaben und simulierten Sensorwerten. So konnte das Konzept bereits vor Erhalt und Einrichten des Raspberry Pi's geprüft und als funktionstauglich bewertet werden.\\
Die ursprüngliche Idee, die Umsetzung mittels automatisierter Tests abzusichern (Langr), musste aus Zeitgründen und fehlenden Erfahrungen fallengelassen werden.\\
Nach Abschluss der Einrichtarbeiten auf dem Raspberry Pi, sind die Tests auf dem Gerät weitergeführt worden. Erst in kurzen Interwallen mit beiziehen von vorhandenen Thermometern zu Hause und später in Langzeittests ist die Software auf ihre Genauigkeit sowie ihre Resilienz gegenüber Abstürzen geprüft worden. Weiter ist auch das Hinzufügen und Entfernen von Sensoren wiederholt getestet worden. Das versenden von E-Mailnachrichten ist mittels bewusstem Heruntersetzen der Grenztemperatur erfolgt. In diesem Bereich bestehen noch Berechtigungsprobleme, welche noch behoben werden, die Auslösung der Nachricht hat in allen Test funktioniert.